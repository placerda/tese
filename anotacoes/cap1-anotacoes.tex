% ## INTRODUCAO

%  Como estruturar o trabalho:
% http://www.biblioteca.fsp.usp.br/~biblioteca/guia/i_cap_02.htm


% \chapter{Introdução} \label{cap:cap_introducao}

% % Deve explicitar os motivos da realização do estudo e destacar sua importância, fornecendo os antecedentes que o justifiquem. Deve conter uma revisão da literatura em que se apresenta a evolução da temática, sua problematização e relevância para o campo da saúde pública, como objeto de investigação. Dependendo da extensão da revisão, ela pode ser destacada da "Introdução", em capítulo à parte denominado "Revisão da literatura" ou "Referencial teórico", quando for o caso. Quando relevante, sobretudo nos casos de doutorado, as hipóteses são destacadas em capítulo à parte, antes dos objetivos.

% A SARS-Cov2 foi declarada uma pandemia global em março de 2020 pela Organização Mundial da Saúde~\cite{world2020director} e desde então um grande número de casos e mortes confirmaram o seu impacto em escala mundial~\cite{WorldHealthOrganization2020CoronavirusReports}. O teste RT-PCR, do inglês \textit{quantitative Reverse-Transcription Polymerase Chain Reaction}, é considerado o padrão ouro para o diagnóstico da COVID-19~\cite{oliveira2020sars}. Estudos mostram que a RT-PCR inicial e a tomografia computadorizada (TC) de tórax apresentam desempenho de diagnóstico comparável na identificação de pacientes suspeitos de COVID-19~\cite{he2020diagnostic} enquanto que o RT-PCR tem desempenho modesto na medida de sensibilidade (53-88\%)~\cite{kovacs2020sensitivity}. A sensibilidade e precisão de nível humano de TC para diagnóstico da COVID-19 são 97\% e 72\%, respectivamente~\cite{caruso2020chest}. 

% Para minimizar o risco potencial de resultados de falso-negativos do teste RT-PCR, a TC de tórax pode ser aplicada em pacientes com suspeita clínica com resultado inicial negativo~\cite{he2020diagnostic}. Vale notar que comparada a radiografia torácica, outro exame de imagem também utilizado para ajudar na detecção da doença, a tomografia computadorizada de tórax é mais sensível que o raio-X para a detecção da COVID-19, bem como para o monitoramento da progressão da doença~\cite{wong2020frequency}.

% % TODO: como a DL ajuda no apoio ao diagnostico da covid na TC.
% Imagens médicas são elementos importantes para o diagnóstico de doenças. O diagnóstico assistido por computador, ou \textit{Computer Aided Diagnostics} (CAD), evoluiu bastante nos últimos anos, juntamente com a capacidade de processamento dos computadores, e o surgimento das técnicas de \textit{deep learning}.  Atualmente a utilização de CNNs aplicada ao diagnóstico já alcança desempenhos equiparáveis a especialistas em radiologia.

% Dado a magnitude dos efeitos da pandemia na sociedade, a pesquisa na área de diagnóstico da COVID-19 auxiliado por computador é de grande importância para desenvolvimento de novos métodos para diagnóstico que possam ser adotados pelos sistemas de saúde de todo o mundo na detecção e tratamento da doença. Embora existam vários estudos com foco no uso de aprendizagem profunda para apoiar o diagnóstico de COVID-19 utilizando tomografias, como descritos no Capítulo \ref{cap:cap_trabalhos}, há espaço para melhorar a sensibilidade da detecção da doença a fim de reduzir ainda mais o número de resultados de teste falso-negativos e, consequentemente, aumentar a chance de detecção e tratamento bem-sucedidos da COVID-19.

% \section{Hipóteses de Pesquisa}\label{sec:obj_perg_hipoteses}

% % Uma forma simples de apresentar a sua pesquisa é dividindo-a em três partes.

% % Eu estou estudando (1. resuma o seu tópico) porque eu quero descobrir (2. identifique o que você não sabe) de modo a ajudar o meu leitor a (3. informe a relevância de sua pesquisa).

% % Portanto:

% % "Estou analisando a cultura X com o objetivo de descobrir se as condições A e B constituem impedimento à produção de alimentos em massa nessa mesma cultura, de modo que a interferência Y possa ser feita de forma menos custosa economicamente para os seus investidores".

% % Assim, a hipótese dessa pesquisa pode ser apresentada da seguinte forma: 

% % "As condições A e B constituem impedimento à produção de alimentos em larga escala na cultura X".

% Este trabalho analisa a utilização de redes neurais convolucionais combinadas com redes neurais recorrentes na detecção da COVID-19 em exames de tomografia computadorizada de tórax. O objetivo é verificar que o desempenho destas redes quando combinadas apresenta um desempenho igual ou superior ao especialista humano ou outras técnicas que aplicam primariamente redes neurais convolucionais para a detecção da COVID-19 em relação a sensibilidade.

% Assim, a hipótese dessa pesquisa pode ser apresentada da seguinte forma: 

% "A aplicação de redes neurais convolucionais combinadas com redes neurais recorrentes constituem uma forma mais eficaz na detecção da COVID-19 em tomografias computadorizadas de tórax quando comparadas com o desempenho humano ou da utilização apenas de um tipo de rede."

% \section{Objetivos do Trabalho}\label{sec:obj_trabalho}

% %  The network that was designed by the method introduced in this paper achieved 82\%, 97\%, and~88\%, respectively, on~precision, sensitivity, and~accuracy for the detection of COVID-19. This demonstrated etter sensitivity than RT-PCR, considered the best test for the detection of COVID-19 in the initial phase of the disease, and~superior performance to CT analysis undertaken by human specialists in specificity and accuracy. In~addition, the~proposed use of hyperparameter optimization improved the baseline in all~metrics.

% O objetivo geral deste trabalho é propor um novo método para detecção da COVID-19 utilizando aprendizagem profunda, através da extração de características espaciais e sequenciais de estudos de tomografia de tórax. 

% Para alcançar o objetivo geral, os seguintes objetivos específicos são estabelecidos: 
% \begin{itemize}
  
%   \item Desenvolver melhorias nas técnicas de preparação de imagens com o objetivo de evidenciar as informações relevantes para o aprendizado das redes neurais relacionado a COVID-19.
  
%   \item Propor técnicas de preprocessamento das imagens para aumento dos dados de forma a melhorar a capacidade de generalização das redes e reduzir o viés por conta do conjunto de dados utilizado.
  
%   \item Propor técnica aplicando redes neurais convolucionais combinadas com redes neurais recorrentes na tarefa de classificar exames de tomografia computadorizada de tórax como COVID-19 ou não COVID-19.
  
%   \item Aplicar técnica de otimização do conjunto de hiperparâmetros a serem empregados para definir a arquitetura e o treinamento das redes neurais utilizadas nesta pesquisa.
  
%   \item Propor técnica de detecção de lesões ocasionadas pela COVID-19 com base em TC, para uso combinado com o classificador da doença, fornecendo mais informações para apoiar o diagnóstico realizado pelo médico.
  
%   \item Avaliar o método proposto através de experimentos usando uma base de imagens públicas de tomografias computadorizadas, a partir de medidas de desempenho comumente utilizadas para avaliar classificadores baseados em aprendizagem profunda.
  
%   \item Comparar o desempenho do método proposto com outros métodos publicados nos últimos 3 anos.
  
% \end{itemize}


% \section{Estrutura do Trabalho}\label{sec:cap_introducao_est_trabalho}

% Esta proposta de tese está organizada da seguinte forma: No  Capítulo \ref{cap:cap_fundamentos}  são apresentados conceitos que são fundamentais para aplicação da técnica proposta, no Capítulo \ref{cap:cap_trabalhos} são comentados alguns métodos publicados recentemente relacionados a aplicação de aprendizagem profunda para diagnóstico da COVID-19, no Capítulo \ref{cap:cap_metodo} é descrito em detalhes o método proposto por este trabalho, no Capítulo \ref{cap:cap_resultados} os resultados dos experimentos são apresentados ao leitor e no Capítulo \ref{cap:cap_discussao} estes resultados são discutidos, interpretas e analisados. Finalmente, o Capítulo \ref{cap:cap_conclusao} apresenta o planejamento, as limitações do estudo até o momento da qualificação, e as contribuições esperadas com a tese de doutorado.

% \section{Notação Utilizada}\label{sec:cap_introducao_notacao}

% TODO: Descrever aqui a notação utilizada para representar vetores $x$, matrizes $X$, elementos $x^{(i)}$, etc.

% % %%%% CAPÍTULO 1 - INTRODUÇÃO
% % %%
% % %% Deve apresentar uma visão global da pesquisa, 
% % %% incluindo: breve histórico, importância e
% % %% justificativa da escolha do tema, delimitações
% % %% do assunto, formulação de hipóteses e objetivos
% % %% da pesquisa e estrutura do trabalho.

% % % Perguntas que podem guiar a introdução - não necessariamente irá ter a resposta para tudo, isso depende da área.
% % % 1 - Qual é o contexto em que seu trabalho está inserido?
% % % 2 - Qual é o problema que motiva a existência deste trabalho?
% % % 3 - Qual é a visão geral da literatura sobre o problema e como é tratado
% % % 4 - Por que a solução na literatura não é o suficiente para ?
% % % 5 - Como seu trabalho trata o problema ?
% % % 6 - como seu trabalho foi avaliado para comprovar que tratou adequadamente o problema?
% % % 7 - De forma geral quais foram os resultados ?
% % % 8 - Quais foram as contribuições do seu trabalho?
% % % 9 -  Como o restante da Dissertação ou Tese está organizada ?


% % \chapter{Introdução}
% % \label{cap:introducao}

% % Deve apresentar uma visão global da pesquisa, incluindo: breve histórico, importância e justificativa da escolha do tema, delimitações do assunto, formulação de hipóteses, objetivos da pesquisa e estrutura do trabalho.


% % Este \'e o primeiro cap\'itulo, apresentamos abaixo as informações de configuração mais utilizadas 



% % \section{Opções de Citação}
% % \label{sec:citacoes}

% % Fonte dos dados sobre citação direta e indireta do site \url{https://www.tecmundo.com.br/tutorial/834-aprenda-a-usar-as-normas-da-abnt-citacao-2-de-4-.htm}, autora \textcite{xavier2020}

% % São comandos de citação deste template usando o pacote abntex2cite:

% % \textbf{textcite\{referencia\}} retorna por exemplo: \textcite{huang2015buffer}

% % \textbf{textcite[n de pag.]\{referencia\}}  retorna por exemplo uma citação com a pagina de referencia: \textcite[p.20]{huang2015buffer}

% % \textbf{cite\{referencia\}} retorna por exemplo: \cite{huang2015buffer}

% % \textbf{cite[n de pag.]\{referencia\}}  retorna por exemplo uma citação com a pagina de referencia: \cite[p.20]{huang2015buffer}

% % Obs: para as aspas ficarem corretas usar no inicio 2 crase `` e no final 2 apóstrofos ''

% % \subsection{Citação direta}

% % Referente as citações diretas existem 2 formas, sendo que os textos devem estar sempre entre aspas (``  texto'') ``a citação direta é a transcrição textual fiel de parte de um conteúdo de uma obra''  \cite{huang2015buffer}. Como os exemplos neste paragrafo onde ``a chamada pelo nome do autor, quando feita no final da citação, deve apresentar-se entre parênteses, contendo o sobrenome do autor em letra maiúscula, seguido pelo ano de publicação e página em que o texto se encontra'' \cite{huang2015buffer}. 

% % Também existe a citação direta quando o o autor está no inicio da citação segundo \textcite{huang2015buffer} ``Assim, o sobrenome do autor deve ser digitado normalmente, com a primeira letra em maiúscula e as demais em minúsculo, seguido do ano e página em que o texto se encontra, sendo estas informações apesentadas entre parênteses''.


% % Se a citação direta for com mais de 03 linhas há uma configuração específica.

% % Conforme descreve \textcite{huang2015buffer}:

% % \begin{quoting}[rightmargin=0cm,leftmargin=4cm] % o comando leftmargin tem o recuo de 4cm
% % \begin{singlespace} %espaço simples 
% % {
% % \footnotesize %comando para a fonte ficar tamanho 10
% % As citações com mais de três linhas devem ter um tipo de destaque diferente: é necessário reduzir o tamanho da fonte, podendo ser para 10 ou 11 e também é preciso aplicar um recuo de 4cm em relação à margem esquerda — selecione o texto e movimente os marcadores, localizado na régua do Word até o número 4, assim, todo o seu texto ficará com o recuo exigido pelas normas (veja a imagem ao lado). Ao final, a citação com mais de três linhas terá a seguinte apresentação — observe que ela não tem aspas.
% % }
% % \end{singlespace} %final do comando de espaço simples
% % \end{quoting} % fim do comando quoting


% % \subsection{Citação Indireta}

% % Escreve com um monte de citações diretas deixa o texto muito carregado de aspas (``  ''). Nesta situação pode-se chegar a conclusão que não se deseja escrever o texto com as palavras exatas do autor, tornando assim o texto mais fluido. Para isso pode usar o texto de lido como base e escrever com suas palavras \cite{huang2015buffer}.


% % Tantos as citações diretas quanto as indiretas podem conter mais de um autor, como por exemplo, \cite{huang2015buffer, huang2015buffer}. Essa forma de citação também pode estar no inicio da conforme \textcite{huang2015buffer, huang2015buffer} as citações pode conter mais de uma referencia no inicio da frase. 



% %  \subsection{O comando \textit{apud}}

% % % ***** O TEXTO DESTA SUBSEÇÃO É DOS AUTORES ABAIXO, PARA DESCREVER O COMANDO apud 
% % % % % % % % % % % % % % % % % % % % % % % % % % % % % % % % % % %
% % %%%% Classe LaTex - MDT-UFSM-2015 - ver. 1.08 (06/09/2017)   %%%%
% % %%%% adaptado da versao de Fabio Natanael Kepler (sem data)  %%%%
% % %%%% Grupo de desenvolvimento:                               %%%%
% % %%%% Franciano Scremin Puhales, Josue Sehnem,                %%%%
% % %%%% Pablo Eli Soares de Oliveira e Diogo Machado Custodio   %%%%
% % %%%% Contato: latexufsm@googlegroups.com                     %%%%
% % % % % % % % % % % % % % % % % % % % % % % % % % % % % % % % % % % 
 
% %   \par A citação \textit{apud} ocorre quando você cita algum autor através de outra obra, sem ter consultado-a propriamente. Neste caso a citação é feita da seguinte forma:
   
% %   A frase original vem de ABC, mas você leu em YYY, não achou o texto original de ABC, então faz um \textit{apud}, neste comando primeiro vem antes vem o autor original (ABC) e depois quem citou (YYY)
  
% %   \begin{center}
% %   \rule{0.5\textwidth}{1pt}\\ 
% %   $\backslash apud\{material\_citado\_no\_material\_lido\}\{material\_lido\}$ \\
% %   \end{center}
% % \begin{verbatim}
% % Sobre a circulação geral da atmosfera pode-se dizer que os ventos do norte
% % não movem moinhos a frase original vem de da2005 mas quem falou foi xavier2020, antes vem o autor original e depois quem citou \apud{da2005}{xavier2020}. 
% % \end{verbatim}
  
% %   Sobre a circulação geral da atmosfera pode-se dizer que os ventos do nortenão movem moinhos \apud{xavier2020}{da2005}.
  
% %   Com citação da pagina de origem \apud[p.5]{xavier2020}{da2005}.
  
% % \begin{center}\rule{0.5\textwidth}{1pt}\end{center}  
% %   \par Nesse caso, na bibliografia só constará a obra consultada e não aquele referenciada pela obra. Para que isso ocorra naturalmente, a obra consultada deve ser incluída normalmente no arquivo referencias.bib enquanto a obra referenciada indiretamente deve ser incluída com a opção \textit{@hidden}, conforme o modelo de referências\footnote{Isto é um teste de nota de rodapé}.

% %       \subsubsection{\textit{Apud on line}}

      
% %       \par O \textit{textapud} se aplica da mesma maneira que o \textit{apud} descrito anteriormente. O termo \textit{on line} é alusivo ao \textit{$\backslash$textcite$\{$label$\}$} definido no abntex. Nesse caso a citação é feita da seguinte forma:
% %       \begin{center}
% %       \rule{0.5\textwidth}{1pt}
% %             $\backslash textapud\{material\_citado\_no\_material\_lido\}\{material\_lido\}$ \\
% % 	    \end{center}

% %  \begin{verbatim}
% % Segundo \textapud{xavier2020}{da2005}, os ventos do
% % norte não movem moinhos.
% % \end{verbatim}

% %             Segundo \textapud{xavier2020}{da2005}, os ventos do norte não movem moinhos.
            
% %             Com citação da pagina de origem  \textapud[p.20]{xavier2020}{da2005}, os ventos do norte não movem moinhos.



% % \section{Siglas e Abreviaturas}
% % \label{sec:siglas}

% % Para usar siglas tem o pacote \acrfull{ACR} que utiliza os comandos:

% % acrfull - \acrfull{ONU} 

% % acrshort - \acrshort{ONU}

% % acrlong -  \acrlong{ONU}



% % \section{Figuras}

% % As figuras podem estar centralizadas ou dependerá da forma de escrita do texto. 

% % Exemplo de Figura: Ver Figura~\ref{fig:exefig}.

% % \begin{figure}[!ht]
% %   \centering
% %   \includegraphics[width=0.3\linewidth]{capitulos/figuras/exefig.eps}
% %   \caption{Exemplo de figura}
% %   \label{fig:exefig}
% % \end{figure}



% % \section{Tabelas}

% % Pode-se utilizar um gerador de tabelas on-line, que agiliza o processo de escrita. Como exemplo temos o site \url{https://www.tablesgenerator.com/}. 


% % Exemplo de Tabela: ver Tabela~\ref{tab:exetab}.


% % \begin{table}[!ht]
% % \begin{center}
% % \caption{Exemplo de tabela}
% % \label{tab:exetab}
% % \begin{tabular}{|c |c |}
% % \hline
% % \textbf{\textbf{Dado 1}} & \textbf{Percentual}\\
% % \hline\hline
% % Tipo 1 & 0,6 \\
% % Tipo 2 & 0,8 \\
% % Tipo 3 & 1,0 \\
% % Tipo 4 & 0,3 \\
% % \hline
% % \end{tabular}
% % \end{center}
% % \end{table}
