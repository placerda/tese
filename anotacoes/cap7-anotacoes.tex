\chapter{Conclusão} \label{cap:cap_conclusao}

% Esta parte apresenta o conjunto das conclusões mais importantes, obrigatoriamente discutidas no texto, respondendo aos objetivos propostos. É uma síntese do que foi defendido na Discussão. As conclusões não devem extrapolar o âmbito dos dados obtidos.

% Após as Conclusões, a critério do autor da tese, podem ser apresentadas "Considerações Gerais" e/ou "Recomendações".

% \section{Comprovação das Hipóteses}\label{sec:cap_conclusao_sec1}

% \section{Trabalhos Futuros}\label{sec:cap_conclusao_sec2}

% \section{Plano de Atividades}\label{sec:cap_conclusao_sec3}

% \section{Produções Científicas}\label{sec:cap_conclusao_sec4}

% Ferramentas, Contribuições Tecnologicas, etc

% %  (discussao)

% % Mesmo com a padronização/normalizaccao das imagens de CT ainda é possível que a AI sofra com a diferenca ente scans de diferentes paises, insituicoes ou instrumentos de CT

% % Estudos futuros: 
% % investicar a integração desses algoritmos na rotina de workflows clinicos para ajudar os radiologistas em acuradamente diagnosticar a void-19.
% % aplicar a técnica em outros datasets par distinguir nao apenas covid-19 de casos normais, mas também covid-19 de outros tipos de doenças como outras pneumonias virais ou bacterianas.

% % classificar a covid de acordo com o nível de severidade (prognóstico).

% % utilizar grad cam para indicar o nível de gravidade (0 - 4)
% % https://arxiv.org/abs/2003.05037

% % Utilizar como referencia o paper machine learing for covid-19: asking the right questions.

% \section{Cronograma}\label{sec:cap_conclusao_cronograma}