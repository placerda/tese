\chapter{Discussão} \label{cap:cap_discussao}

Durante a otimização de hiperparâmetros os resultados obtidos com os melhores hiperparâmetros da Rede CNN são 84\%, 98\% e 90\% respectivamente em precisão sensibilidade e F1 Score, enquanto que os resultados com os melhores hiperparâmetros da Rede Combinada (CNN e LSTM) foram 90\%, 97\% e 94\% mostrando um resultado melhor em precisão e F1 Score e a sensibilidade similiar quando comparada a rede CNN. Os resultados obtidos durante o processo de HPO com a introdução da rede neural recorrente (LSTM) mostrou desempenho melhor que os resultados obtidos com a rede neural convolucional apenas, mostrando um potencial de ganho em aplicar o método proposto. 

A rede neural convolucional projetada pelo método apresentado neste artigo alcançou 79\%, 98\% e ~88\%, respectivamente, em precisão, sensibilidade e acurácia para a detecção de COVID-19 na validação interna cruzada. Esse resultado mostrou melhor sensibilidade do que o RT-PCR, considerado o melhor teste para a detecção de COVID-19 na fase inicial da doença e desempenho superior à análise de TC realizada por especialistas humanos em especificidade e precisão. 

Os resultados obtidos na validação externa mostram que o método precisa ser melhorado para proporcionar bons resultados com o conjunto de dados externos.

